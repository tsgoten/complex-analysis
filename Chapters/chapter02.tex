\chapter{Complex Numbers}
\section{Sums and Products}
Complex numbers can be represented as ordered pairs $ (x,y) $ where the complex number is given by $ z = x + iy $. Two complex numbers are said to be equal iff their respective real and imaginary parts are equal 
\begin{example}
	\begin{align}
		z = 4 + 3i \\
		w = 5 + 3i \\
		z \neq w
	\end{align}
\end{example}
The sum of two complex numbers is defined by the sum of the respective real and imaginary parts. That is 
\begin{example}
	\begin{align}
		z = (x,y) \indent w = (u,v) \\
		z + w = (x+u, y+v)
	\end{align}
\end{example}
\section{Basic Algebraic Properties}
Most of the properties for addition and multiplication are same for complex numbers \\
The communitive properties still hold
\begin{align}
z +w = w+z \\ 
zw = wz
\end{align}
Similarly, the associative and distributive properties hold as well. This will be verified in the exercises section.   
\section{Vectors and Moduli}
Complex numbers can be represented as a vector from the origin to the point $ (x,y) $. Consider the figure below showing the point $ -2 + i $
\putpiclinewidth{23}
Treating complex numbers as vectors allows us to take their distance from the origin, or the modulus. Denoted by $ |z| $. We can now compare two complex numbers by their modulus. \\
The modulus can be computed using Pythagorean theorem.
\begin{example}
	Describe the figure represented by the equation $ |z -1 + 3i| = 2 $ \\ This represents a circle whose center $ z_0 = (1,-3) $ and whose radius is $ R =2 $
\end{example} 
\section{Complex Conjugates}
The complex conjugate or simply the conjugate of a complex number \[ z = x + iy \] is define by \[ \bar{z} = x - iy \] and denoted by $ \bar{z} $. Considering the ordered pairs it is apparent that the conjugate is a reflection of the complex number over the real axis. 

\section{Exercises}
\begin{exercise}
	Show that complex numbers are distributive and associative
\end{exercise}
\begin{exercise}
	Show the following
	\begin{enumerate}
		\item $ \overline{z_1 + z_2} = \overline{z_1} + \overline{z_2}$
		\item derive the modulus of a complex number 
		\item $ \Re z = \dfrac{z + \bar{z}}{2} $
		\item $ \Im z = \dfrac{z - \bar{z}}{2i} $
		\item $ z^{-1} $
	\end{enumerate}
\end{exercise}